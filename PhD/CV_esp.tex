%!TEX TS-program = xelatex
\documentclass[]{friggeri-cv}
\usepackage[utf8]{inputenc}

\begin{document}
\header{\hspace{25 mm}Manuel Razo-Mejia}{}
       {\hspace{40 mm}Estudiante de Posgrado - Bioqu\'imica \& Biof\'isica Molecular. Caltech}


% In the aside, each new line forces a line break
\begin{aside}
	\section{contacto}
		1036 E. Del Mar Blvd.\\
		apt 102, 91106\\
		Pasadena, CA.
  		mrazomej at caltech dot edu\\
  		(626) 590 3634
	
	\section{intereses}
	f\'isica biol\'ogica\\biologia de sistemas\\biolog\'ia computacional\\evoluc\'ion
	
	\section{idiomas}
		ingl\'es flu\'ido\\ 
		TOEFL iBT 116/120
    
    	\section{habilidades de programaci\'on}
    		R (bioconductor)\\python\\matlab\\mathematica\\shell scripting
\end{aside}


\section{educaci\'on}
	\textbf{PhD Biochemistry \& Molecular Biophysics (2014-presente)}\\
	California Institute of Technology, Pasadena, CA\\
    	\textbf{Licenciatura Ingenier\'ia Biotecnol\'ogica (2009-2014)}\\
    	Instituto Polit\'ecnico Nacional, Silao, Guanajuato, M\'exico.\\
	{\small\addfontfeature{Color=lightgray}Cursos relacionados: Bioingenier\'ia, termodin\'amica, biolog\'ia molecular,  procesos de separaci\'on, f\'isica general, ecuaciones diferenciales, c\'alculo vectorial, estad\'istica \& probabilidad.}}\\

\section{experiencia de investigaci\'on}
{\small\addfontfeature{Color=lightgray}Junio - Agosto 2013}}\\
\textbf{Weizmann Institute of Science, Rehovot, Israel.}\\
Asesor: \textbf{Ron Milo},  Profesor Asistente.\\
Ingenier\'ia metab\'olica y biolog\'ia sint\'etica para incrementar el crecimiento natural de la bacteria \textit{E. coli} en glioxylato como parte de un esfuerzo m\'as general para dise\~nar un organismo autotr\'oficio a partir de un chasis heterotr\'ofico.\\
{\small\addfontfeature{Color=lightgray}Habilidades adquiridas: Ensamblaje de bibliotecas g\'enicas, screening fenot\'ipico automatizado de alto rendimiento, principios de ingenier\'ia metab\'olica.}\\

{\small\addfontfeature{Color=lightgray}June 2012 - June 2013}}\\
\textbf{California Institute of Technology, Pasadena, CA.}\\
Asesor: \textbf{Rob Phillips}, Profesor Fred y Nancy Morris de Biof\'isica y Biolog\'ia.\\
Implementaci\'on de un modelo termodin\'amico de regulaci\'on gen\'etica en el oper\'on lac para explorar la variabilidad de la regulaci\'on del circuito; logrando un mapeo entre cambios en las bases del ADN a cambios en los par\'ametros del modelo y prediciendo el cambio fenot\'ipico correspondiente. Al analizar aislados naturales de \textit{E. coli} de todo el mundo pudimos explorar la variabilidad natural de este circuito gen\'etico entre diferentes miembros de la misma especie.\\
{\small\addfontfeature{Color=lightgray}Habilidades adquiridas: Modelado biof\'isico basado en mec\'anica estad\'istica,  biolo\'ia molecular de \textit{E. coli}.}\\

{\small\addfontfeature{Color=lightgray}2011/2013}\\
\textbf{LANGEBIO-CINVESTAV, Irapuato, M\'exico.}\\
Asesor: \textbf{Cei Abreu-Goodger}, Profesor Asistente.\\
An\'alisis computacional de la conservaci\'on de blancos de microRNAs entre pez zebra y rat\'on usando m\'etodos estad\'isticos y bioinform\'aticos.\\
{\small\addfontfeature{Color=lightgray}Habilidades adquiridas: uso de sistemas operativos basados en UNIX, herramientas bioinform\'aticas como bioconductor, habilidades de programaci\'on en general.}\\

{\small\addfontfeature{Color=lightgray}Diciembre 2011 - Enero 2012}\\
\textbf{Department of Systems Biology, Harvard Medical School, Boston, MA.}\\
Asesor: \textbf{Michael Springer}, Profesor Asistente.\\
Desarrollo de un m\'todo basado en amplificaciones de PCR para la identificaci\'on de 64 aislados naturales de \textit{S. cerevisiae} basado en su perfil de mutaciones.\\
{\small\addfontfeature{Color=lightgray}Habilidades adquiridas: Biolog\'ia molecular de levadura.}\\

\section{publicaciones cient\'ificas}
\textbf{Razo-Mejia, M.}, Boedicker, J. Q., Jones, D., DeLuna, a, Kinney, J. B., & Phillips, R. (2014). \textit{Comparison of the theoretical and real-world evolutionary potential of a genetic circuit}. Physical Biology, 11(2), 026005. doi:10.1088/1478-3975/11/2/026005\\

Lior Zelcbuch, \textbf{Manuel Razo-Mejia}, Elad Herz, Sagit Yahav, Niv Antonovsky, Hagar Kroytoro, Ron Milo, Arren Bar-Even. (en revisi\'on). \textit{An in vivo metabolic approach for deciphering the product specificity of glycerate kinase proves that both E. coli's glycerate kinases generate 2-phosphoglycerate}.


\section{premios/becas}
Benjamin M. Rosen Graduate Fellowship\\
{\addfontfeature{Color=lightgray}California Institute of Technology, 2014}\\
\\
Mejor promedio Ingenier\'ia Biotecnol\'ogica,  2014\\
{\addfontfeature{Color=lightgray}Instituto Polit\'ecnico Nacional, 2014}\\
\\
Summer Kupcinet-Getz International Science School Fellowship\\
{\addfontfeature{Color=lightgray}Weizmann Institute of Science, 2013}\\
\\
Summer Undergraduate Research Fellowship\\
{\addfontfeature{Color=lightgray}California Institute of Technology, 2012}\\

\section{experiencia docente}
Instructor - ``Biolog\'ia a trav\'es de los n\'umeros".\\
Desscripci\'on: Curso intensivo de biof\'isica para introducir a estudiantes de \'ultimo a\~no de preparatoria y de 1er o 2ndo a\~no de licenciatura a los temas actuales en biolog\'ia cuantitativa.\\
Duraci\'on: 1 semana, 10-12 horas al d\'ia\\
{\addfontfeature{Color=lightgray}Clubes de Ciencia Mexico. Ensenada Baja California, 2014}\\

Asistente - Bi1X: Las grandes ideas de la biolog\'ia, una introducci\'on a trav\'es de la experimentaci\'on.\\
Descripci\'on: Bi1X provee a los estudiantes una introducci\'on a los conceptos y m\'etodos de laboratorio en biolog\'ia. Biolog\'ia molecular y microscop\'ia avanzada fueron combinadas para explorar las grandes ideas de la biolog\'ia.\\
Duraci\'on: 10 semanas, 6 horas a la semana.\\
{\addfontfeature{Color=lightgray}California Institute of Technology, 2013}\\

Instructor - Introducci\'on a la Bioinform\'atica.\\
Descripci\'on: Curso extracurricular para aprender los conceptos b\'asicos y habilidades de programaci\'on en R \& Bioconductor para el an\'alisis de microarreglos y datos de secuenciaci\'on.\\
Duraci\'on: 5 semanas, 4 horas a la semana.\\
{\addfontfeature{Color=lightgray}Instituto Polit\'ecnico Nacional, 2011}\\


\section{actividades extracurriculares}
\textbf{Cubes de Ciencia Mexico. Comit\'e de organizaci\'on.}\\
La emergente asociaci\'on sin fines de lucro Clubes de Ciencia M\'exico tiene como objetivo expandir el acceso a j\'ovenes estudiantes Mexicanos a eduaci\'on cient\'ificia de calidad. Para esto dise\~namos e implementamos cursos de ciencia, tecnolog\'ia, ingenier\'ia y matem\'aticas para estudiantes de preparatoria y licenciatura. Nuestros clubes de ciencia son un mecanismo para establecer una red de mentores que conecte a prominentes cient\'ificios en M\'exico y en el extranjero con estudiantes Mexicanos interesados en la ciencia. Esta red internacional de mentores intenta canalizar 3 objetivos principales:\\
(1) Incrementar el inter\'es de lis estudiantes por la ciencia.\\
(2) Guiar a los estudiantes hacia carreras cient\'ificas.\\
(3) Desarrollar habilidades t\'ecnicas y cognitivas relacionadas con la ciencia.

\end{document}
