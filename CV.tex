%!TEX TS-program = xelatex
\documentclass[]{friggeri-cv}

\begin{document}
\header{\hspace{25 mm}Manuel Razo-Mejia}{}
       {\hspace{40 mm}PhD Candidate - Biochemistry \& Molecular Biophysics. Caltech}


% In the aside, each new line forces a line break
\begin{aside}
	\section{contact}
		1200 E. California Blvd.\\
		91125 MC 114-96\\
		Pasadena, CA.\\
  		mrazomej \{at\} caltech \{dot\} edu\\
  		(626) 590 3634\\
		Citizenship: Mexican\\
	
	\section{interests}
	physical biology\\
	evolution\\
	systems biology\\
	computational biology\\
	
	\section{languages}
		spanish-native\\
		english-fluent\\ 
		TOEFL iBT 116/120
    
    	\section{programming skills}
		fluent in \textbf{Python}\\
    		matlab\\
		mathematica\\
		shell scripting
\end{aside}


\section{education}
	\textbf{PhD Biochemistry \& Molecular Biophysics (2014-present)}\\
	Advisor: Rob Phillips.\\
	California Institute of Technology, Pasadena, CA\\
	\textbf{Physiology course (Summer 2015)}\\
	Marine Biological Laboratory, Woods Hole, MA\\
    	\textbf{BSc Biotechnological Engineering (2009-2014)}\\
    	Instituto Politecnico Nacional, Silao, Guanajuato, Mexico.\\

\section{current research}

The focus of my PhD research is to ask whether or not information - in its mathematical definition - can be a quantitative trait on which natural selection can act. To tackle this question we make use of statistical mechanics, information theory and population genetics, along with laboratory precision measurements of biological systems. In particular we study how the physics of gene regulation can affect the evolutionary trajectory of a population. Given our understanding of how genes are turned on and off in bacteria we aim to infer the information processing capacity of these genetic circuits. We use this inference to predict how having access to more bits of information can ensure that a particular bacterium will have higher fitness. We see this exciting direction as part of the overall effort in quantitative biology of predicting evolution.

\section{publications}
\begin{flushleft}
Chure, G., \textbf{Razo-Mejia, M.}, Belliveau, N. M., Einav, T., Kaczmarek, Z. A., & Barnes, S. L. (2019). \textit{The Energetics of Molecular Adaptation in Transcriptional Regulation}. bioRxiv.\linebreak
http://doi.org/10.1101/638270\\
\end{flushleft}

\begin{flushleft}
\textbf{Razo-Mejia, M.}, & Phillips, R. (2019). \textit{First-principles prediction of the information processing capacity of a simple genetic circuit}. bioRxiv, (1), 594325.\linebreak
http://doi.org/10.1101/594325
\end{flushleft}\\

\begin{flushleft}
Foster, P. J., \textbf{Razo-Mejia, M.}, & Phillips, R. (2019). \textit{Measuring the Energetic Costs of Embryonic Development}. Developmental Cell, 48(5), 591?592.\linebreak
http://doi.org/10.1016/j.devcel.2019.02.016
\end{flushleft}\\

\begin{flushleft}
Phillips, R., Belliveau, N. M., Chure, G., Garcia, H. G., \textbf{Razo-Mejia, M.}, & Scholes, C. (2019). \textit{Figure 1 Theory Meets Figure 2 Experiments in the Study of Gene Expression}. Annual Review of Biophysics, 48(1), 121?163.\linebreak
http://doi.org/10.1146/annurev-biophys-052118-115525
\end{flushleft}\\

\begin{flushleft}
\textbf{Razo-Mejia, M.}*, Barnes, S. L.*, Belliveau, N. M.*, Chure, G.,* Einav, T.*, Lewis, M., \& Phillips, R. (2018). \textit{Tuning Transcriptional Regulation through Signaling : A Predictive Theory of Allosteric Induction}. Cell Systems, 1-14. * These authors contribute equally. \linebreak
http://doi.org/10.1016/j.cels.2018.02.004.
\end{flushleft}\\

\begin{flushleft}
Lior Zelcbuch, \textbf{Manuel Razo-Mejia}, Elad Herz, Sagit Yahav, Niv Antonovsky, Hagar Kroytoro, Ron Milo, Arren Bar-Even. (2015). \textit{An in vivo metabolic approach for deciphering the product specificity of glycerate kinase proves that both E. coli's glycerate kinases generate 2-phosphoglycerate}. Plos One, 10(3), e0122957.\linebreak
http://doi.org/10.1371/journal.pone.0122957
\end{flushleft}\\

\textbf{Razo-Mejia, M.}, Boedicker, J. Q., Jones, D., DeLuna, a, Kinney, J. B., \& Phillips, R. (2014). \textit{Comparison of the theoretical and real-world evolutionary potential of a genetic circuit}. Physical Biology, 11(2), 026005.\linebreak
http://doi:10.1088/1478-3975/11/2/026005\\

\pagebreak

\section{awards/scholarships}

Caltech's Breakthrough campaign featured story\\
{\addfontfeature{Color=lightgray}California Institute of Technology, 2018}\\
Link: \url{https://breakthrough.caltech.edu/story/science-and-serendipity/}\\
\\
Amgen Graduate Fellowship\\
{\addfontfeature{Color=lightgray}Caltech-Amgen Research Collaboration, 2015}\\
\\
Benjamin M. Rosen Graduate Fellowship\\
{\addfontfeature{Color=lightgray}California Institute of Technology, 2014}\\
\\
Valedictorian, Class of 2014\\
{\addfontfeature{Color=lightgray}Instituto Politecnico Nacional, 2014}\\
\\
Summer Kupcinet-Getz International Science School Fellowship\\
{\addfontfeature{Color=lightgray}Weizmann Institute of Science, 2013}\\
\\
Summer Undergraduate Research Fellowship\\
{\addfontfeature{Color=lightgray}California Institute of Technology, 2012}\\


\section{conferences, talks, poster presentations}

{\small\addfontfeature{Color=lightgray} 2019}\\
\textbf{"Stochastic Physics in Biology" Gordon Research Conference}.\\
{\addfontfeature{Color=lightgray}Invited speaker.}\\
Title: Does evolution care about bits? Information as currency in nature's stock market.

{\small\addfontfeature{Color=lightgray} 2018}\\
\textbf{American Society for Cell Biology}.\\
{\addfontfeature{Color=lightgray}Poster presentation.}\\
Title: The physics of information processing of a simple genetic circuit.

{\small\addfontfeature{Color=lightgray} 2017 \& 2018 }\\
\textbf{Cell Modeling Hackathon Workshop.}\\
{\addfontfeature{Color=lightgray}QBCnet.}\\
Three-day workshop that brings theorists and experimentalists together to work on generating theoretical descriptions of experimental observations. The intention of the short workshop is to promote the dialogue between different approaches in biology in order to encourage new collaborations.

\section{teaching experience}

Teaching assistant - Physical Biology of the Cell @ NCBS Bangalore.\\
Description: This intensive week long mini-course explores the way that physical and mathematical models can be used to understand biological systems. The course begins by examining the way in which simple order of magnitude estimates can be used to provide insights into problems ranging from the fidelity of protein translation to how far a bird can fly without stopping to how amphibians arrive on oceanic islands. This is followed by the use of statistical mechanics to explore problems in regulatory biology. Some examples include the physics of post-translational modifications, how cells make transcriptional decisions and the precision with which embryonic development takes place. In addition to these topics, the course also involves a series of hands-on projects using Python to amplify the case studies discussed in class.
Year: 2019.\\
Duration: 1 week.\\
{\addfontfeature{Color=lightgray}National Center for Biological Sciences, Bangalore, India}\\

Teaching assistant - Bi/Ge105: Evolution.\\
Description: The objective of this course is to use broad brush strokes to paint a picture of modern thinking on evolution, both of living organisms and the planet they inhabit. The first part of the course takes a decidedly historical and naturalistic perspective, focusing on the timeline of evolution (based on the geological record of fossils and geochemical signatures) and how the natural history of both the inanimate and animate worlds in the hands of Darwin and Wallace (and their distinguished predecessors) led to the articulation of the tenets of evolution by natural selection as well as a picture of a dynamic Earth. This is followed by an examination of the emergence of modern genetics which gave us a molecular picture of variation. Next, we undertake a study of the forces of evolution such as selection, drift, migration and mutation and how population genetics provides a quantitative framework for examining the interplay between these forces. With these preliminaries in hand, we will then turn to a variety of case studies in evolution that illustrate the principles of the subject with some of the remarkable studies that have been made to test these ideas.\\
Year: 2018.\\
Duration: 10 weeks.\\
{\addfontfeature{Color=lightgray}California Institute of Technology}\\

Teaching assistant - Physiology Course at the MBL.\\
Description: The MBL Physiology Course is one of the oldest continually running biology courses in the world. The Course traditionally has had three goals that we strongly endorse: graduate training, cutting edge research, and introducing new generations of scientists to the unique environment of the MBL. The modern vision is a Course that brings together biological and physical / computational scientists, both in the faculty and the student body, to work together on cutting edge problems in cell physiology. These interactions create an environment that is more of a summer school in interdisciplinary science than a conventional graduate course.\\
Years: 2016, 2017 \& 2018.\\
Duration: 3 weeks each.\\
{\addfontfeature{Color=lightgray}Marine Biological Laboratory at Woods Hole, MA.}\\

Teaching assistant - Physical Biology Course at the MBL.\\
Description: The course explores the description of a broad array of topics from modern biology using the language of physics and mathematics. It focuses on physical and mathematical model building by drawing examples from broad swaths of modern biology including cell biology (signaling and regulation, cell motility), physiology (metabolism, swimming and flight), developmental biology (patterning of body plans, how size and number of organelles and tissues are controlled), neuroscience (action potentials and ion channel gating, vision) and evolution (population genetics, biogeography).\\
Year: 2016, 2017 \& 2018.\\
Duration: 3 weeks each.\\
{\addfontfeature{Color=lightgray}Marine Biological Laboratory at Woods Hole, MA.}\\

Teaching assistant - BE/Bi103: Data Analysis in the Biological Sciences.\\
Description: Modern biology is a quantitative science, and biologists need to be equipped with tools to analyze quantitative data. This course takes a hands-on approach to developing these tools. Among other techniques, we show how to do regression, parameter estimation, outlier detection and correction, error estimation, hypothesis testing, denoising, and image processing and quantification.\\
Years: 2015 / 2016.\\
Duration: 10 weeks each.\\
{\addfontfeature{Color=lightgray}California Institute of Technology}\\

Teaching assistant - BE/Bi101: Order of Magnitude Biology.\\
Description: Students develop skills in the art of educated guesswork and apply them to the biological sciences. Building from a few key numbers in biology, students will ?size up? biological systems by making inferences and generating hypotheses about phenomena such as the rates and energy budgets of key biological processes. The course will cover the breadth of biological scales: molecular, cellular, organismal, communal, and planetary.\\
Year: 2015.\\
Duration: 10 weeks.\\
{\addfontfeature{Color=lightgray}California Institute of Technology.}\\

Teaching assistant - Evolution @ GIST.\\
Description: The objective of this course is to use broad brush strokes to paint a picture of modern thinking on evolution, both of living organisms and the planet they inhabit.\\
Years: 2015 / 2018.\\
Duration: 10 days.\\
{\addfontfeature{Color=lightgray}Gwangju Institute of Science and Technology, Gwangju, South Korea.}\\


Instructor - ``Biolog\'{i}a a trav\'{e}s de los n\'{u}meros" (biology by the numbers).\\
Description: Intense biophysics bootcamp to introduce senior high-school students, 1st \& 2nd year undergraduate students to the current challenges in quantitative biology.\\
Year: 2014.\\
Duration: 1 week.\\
{\addfontfeature{Color=lightgray}Clubes de Ciencia Mexico. Ensenada Baja California.}\\

Teaching assistant - Bi1X: The great ideas of biology, an introduction through experimentation.\\
Description: Bi1x provides students with an introduction to concepts and laboratory methods in biology. Molecular biology techniques and advanced microscopy were combined to explore the great ideas of biology.\\
Year: 2013.\\
Duration: 10 weeks.\\
{\addfontfeature{Color=lightgray}California Institute of Technology.}\\

\section{extracurricular activities}
\textbf{Cubes de Ciencia Mexico. Organization committee}\\
The emerging non-profit association Clubes de Ciencia Mexico aims to expand the access of young Mexican students to high quality scientific education. For this we design and implement science, technology, engineering and math workshops for high school students and freshman undergrads. Our science clubs are the mechanism to establish a network of mentors that link the most prominent young scientists in Mexico and abroad with other Mexican students interested in science. This international network of mentors tries to catalyze three main objectives:\\
(1) Increase the interest for science among the students.\\
(2) Guide young students towards scientific careers.\\
(3) Develop science-related technical and cognitive abilities.

\newpage 
\section{personal references}
\begin{itemize}
	\item \textbf{Professor Rob Phillips}\\
	Phone:  +1 (626) 395-3374\\
	Email:  phillips \{at\} pboc \{dot\} caltech \{dot\} edu\\
	Address:  California Institute of Technology\\
	1200 E. California Blvd\\
	Pasadena, CA 91125
	
	\item \textbf{Professor Dianne Newman}\\
	Phone:  +1 (626)  395-3543\\
	Email:  dkn \{at\} caltech \{dot\} edu\\
	Address:  California Institute of Technology\\
	1200 E. California Blvd\\
	Pasadena, CA 91125
	
	\item \textbf{Professor Ron Milo}\\
	Phone: +9 (728) 934-4466\\
	Email:  ron.milo \{at\} weizmann \{dot\} ac \{dot\} il\\
	Address:  Weizmann Institute of Science\\
	Herzl St 234\\
	Rehovot, 7610001, Israel
\end{itemize}
\end{document}
